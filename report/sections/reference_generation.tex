\section{Reference Generation}

\subsection{Sine Wave Generation}
In this exercise, a specific sinusoidal reference signal was created to test the system. The requirements were an offset of $45^{\circ}$, an amplitude of $20^{\circ}$, and a frequency of $5$ rad/s[cite: 38].

The following MATLAB function was implemented to generate this signal:

\begin{lstlisting}[caption={MATLAB function for Reference Generation}, language=Matlab]
function signal = ReferenceGenerator(t, A, w, phase)
    % Generate signal with Amplitude A, frequency w, and phase offset
    signal = A * sin(w*t + deg2rad(phase));   
end

% Parameters used in Lab 3
A = 20;                 % Amplitude 
w = 5;                  % Frequency (rad/s)
phase = 45;             % Offset
\end{lstlisting}

\subsection{Time Response Analysis}
The system was excited with the generated signal, and the actual position was recorded. The goal was to plot the reference position versus the actual position to evaluate the sensor performance and tracking accuracy[cite: 41].

The plotting process utilized the following code:

\begin{lstlisting}[caption={Code for plotting Time Response}, language=Matlab]
% Load data
[desired_pos, actual_pos, t_sim] = load_recording('Lab3_2.1.txt');

% Plot Reference vs Actual
figure;
subplot(2,1,1); 
plot(t_sim, desired_pos, 'LineWidth', 1.5); hold on;
plot(t_sim, actual_pos, 'LineWidth', 1.5);
legend('Reference', 'Actual');
xlabel('Time [s]'); ylabel('Position [deg]');

% Plot Error
subplot(2,1,2); 
plot(t_sim, desired_pos - actual_pos);
ylabel('Error [deg]');
\end{lstlisting}

Figure \ref{fig:ex2_response} shows the resulting response. The actual position (orange) follows the general shape of the reference (blue), but significant artifacts are present where the position drops sharply. The position error peaks between $20^{\circ}$ and $40^{\circ}$, indicating that while the sensor is functional, the tracking performance suffers from mechanical or control-related disturbances.

\begin{figure}[H]
    \centering
    \includegraphics[width=0.85\textwidth]{img/Ex2_Response.png}
    \caption{Reference vs. Actual Position (Top) and Position Error (Bottom).}
    \label{fig:ex2_response}
\end{figure}

\section{System Identification in the Frequency Domain}
The exoskeleton was approximated as a linear system $G(s)$[cite: 59]. The identification process involved exciting the system at different frequencies to extract the frequency response[cite: 62].

\subsection{Experimental Bode Plot}
The system was excited with frequencies ranging from 0.5 to 9 rad/s[cite: 94]. For each recording, the magnitude ratio (gain) and time delay (phase shift) were extracted automatically to construct the experimental Bode plot.

\begin{lstlisting}[caption={Automated Parameter Extraction for Bode Plot}, language=Matlab]
% Loop through recorded files for different frequencies
for i = 1:length(files)
    [des, act, t_s] = load_recording(files{i});
    
    % 1. Estimate Frequency (w) via zero-crossings/peaks
    % ... (Peak detection code) ...
    w_exp(i) = w_val;
    
    % 2. Estimate Gain (Ratio of RMS amplitudes)
    amp_in = rms(des_ac) * sqrt(2);
    amp_out = rms(act_ac) * sqrt(2);
    gain_exp(i) = amp_out / amp_in;
    
    % 3. Estimate Phase (Time delay via cross-correlation)
    [acor, lag] = xcorr(act_ac, des_ac);
    timeDiff = lag(I) * mean(diff(t_s));
    phase_exp(i) = -w_val * timeDiff * (180/pi); 
end

% Plotting the Experimental Bode
semilogx(rads2hz*w_exp, 20*log10(gain_exp), 'r*-');
semilogx(rads2hz*w_exp, phase_exp, 'r*-');
\end{lstlisting}

As seen in Figure \ref{fig:bode_exp}, the magnitude response is unexpectedly high ($>12$ dB), suggesting the output amplitude exceeded the input significantly. The phase response shows a characteristic lag that increases with frequency.

\begin{figure}[H]
    \centering
    \includegraphics[width=0.85\textwidth]{img/Ex3_1_ExperimentalBode.png}
    \caption{Experimental Bode diagram.}
    \label{fig:bode_exp}
\end{figure}

\subsection{Model Fitting}
A first-order transfer function model was proposed to fit the experimental data[cite: 113]:
\begin{equation}
    G(s) = \frac{K}{\tau s + 1}
\end{equation}
The theoretical parameters chosen were $K=1$ and $\tau=0.1$ s.

\begin{lstlisting}[caption={Model Fitting Code}, language=Matlab]
% Theoretical parameters
K = 1;
tau = 0.1;
num = K; den = [tau, 1];

% Generate Theoretical Bode
w_theoretical = logspace(log10(0.1), log10(100), 100);
[gain_theo, phase_theo] = bode(num, den, w_theoretical);

% Compare with Experimental Data
semilogx(rads2hz*w_exp, 20*log10(gain_exp), 'r*', ...
         rads2hz*w_theoretical, 20*log10(gain_theo), 'b-');
\end{lstlisting}

Figure \ref{fig:model_fit} compares the model with the experimental data. The theoretical DC gain ($K=1 \to 0$ dB) is much lower than the experimental gain, indicating $K$ must be increased. The phase trend is similar, though the experimental system exhibits a sharper phase drop, hinting at potential time delays not captured by a simple first-order model.

\begin{figure}[H]
    \centering
    \includegraphics[width=0.8\textwidth]{img/Ex3_2_ModelFitting.png}
    \caption{Comparison of Experimental Data and Theoretical Model ($K=1, \tau=0.1$).}
    \label{fig:model_fit}
\end{figure}

\subsection{Bandwidth Analysis}
The bandwidth is defined as the frequency where the magnitude drops 3 dB below the DC value[cite: 65]. For the estimated first-order system ($\tau=0.1$ s), the bandwidth is calculated as:

\begin{lstlisting}[caption={Bandwidth Calculation}, language=Matlab]
% Bandwidth for 1st order system is 1/tau [rad/s]
bw_rad = 1/tau;         % 10 rad/s
bw_hz = bw_rad / (2*pi); % ~1.59 Hz
\end{lstlisting}

The calculated bandwidth is \textbf{1.59 Hz}. 
\textbf{Analysis:} While this covers the primary frequency range of normal human motion ($<1$ Hz), it is on the lower side for high-performance assistive devices. A higher bandwidth (e.g., 4-8 Hz) would be preferable to handle faster reflex movements or tremor suppression effectively without inducing noticeable lag for the user.